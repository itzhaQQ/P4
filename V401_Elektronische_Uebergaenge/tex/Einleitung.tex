\chapter{Einleitung}
In Versuch \textit{401: Elektronische Übergängen in Atomen} werden elektronische Übergänge zwischen diskreten Energieniveaus in Atomen anhand zweier experimenteller
Methoden untersucht.
\vspace{0.3cm}\\
Im ersten Versuchsteil wird die Aufspaltung atomarer Spektrallinien im externen Magnetfeld, der \textit{Zeemann}-Effekt, betrachtet. Anhand eines ausgewählten
Übergangs im Cadmium (Cd)-Atom ($5^1D_2 \rightarrow 5^1P_1$) wird die Abhängigkeit der Linienaufspaltung von der Magnetfeldstärke sowie die Polarisationseigenschaften der 
emmittierten Strahlung untersucht. Zur hochauflösenden Darstellung werden ein \texttt{Fabry-Pérot-Etalon} und eine \texttt{CCD-Kamera} verwendet.\\
Im Rahmen der Auswertung werden das \textit{Bohrsche-Magneton} sowie die Finesse und das Auflösungsvermögen des Etalons bestimmt.
\vspace{0.3cm}\\
Der zweite Versuchsteil befasst sicht mit dem  \textit{Franck-Hertz}-Versuch, der die diskrete Energieaufnahme von Atomen bei Elektronenstößen untersucht.\\
Durch Aufnahme der Anodenstrom-Spannungs-Kurve und deren Analyse wird die Anregungsenergie von Quecksilber (Hg)-Atomen ermittelt.\\
Zusätzlich werden die Einflüsse von Temperaturänderung und variabler Gegenspannung auf die Form der Stromkurve untersucht, um die Wechselwirkung zwischen freien Elektronen und Hg-Atomen
unter verschiedenen Bedingungen zu analysieren.
\vspace{0.3cm}\\
Der gesamte Versuch dient dazu, die Quantelung atomarer Eigenschaften experimentell nachzuweisen und daraus charakteristische Größen herzuleiten.