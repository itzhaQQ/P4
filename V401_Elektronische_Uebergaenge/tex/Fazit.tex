\chapter{Fazit}

Dieser Versuch handelte von den elektronischen Übergängen in Atomen und gliedert sich in zwei Hauptteile. Der erste Hauptteil beschäftigt sich mit dem Zeeman-Effekt, während der zweite Teil den Franck-Hertz-Versuch behandelt.
\vspace{0.3cm}\\
Im ersten Versuchsteil konnten wir zunächst optisch mit Hilfe eines Fabry-Pérot-Etalons das Licht einer Cadmium-Lampe abhängig von der Wellenlänge auflösen und mit einem Okular und Polarisationsfiltern die Energieverschiebung durch den Zeeman-Effekt sowohl in transversaler als auch in longitudinaler Richtung beobachten und nachweisen. Im weiteren Versuchsverlauf konnten wir die Energieverschiebung quantitativ mit Hilfe einer CCD-Kamera berechnen und die Abhängigkeit der Energieverschiebung von dem anliegenden Magnetfeld analysieren und darüber das Bohrsche Magneton $\mu_B$ bestimmen. Unsere beiden Werte für $\sigma_+$- und $\sigma_-$-polarisiertes Licht haben eine große Abweichung untereinander gezeigt: $\mu_{\sigma_+}$ = $(\num{9.73e-24}\pm\num{1.61e-12}) \si{\joule\per\tesla}$ und $\mu_{\sigma_-}$ = $(\num{24.43e-24}\pm\num{1.36e-12}) \si{\joule\per\tesla}$. Der Literaturwert hingegen beträgt: $\mu_B$ = $9{,}2740100783(29)\times10^{-24}\,\mathrm{J/T}$. 
\vspace{0.3cm}\\
In weitgehenden Überlegungen konnten für die Finesse $\mathcal{F}$ = $(4,383 \pm 0,045 )$ und für das Auflösungsvermögen $A = (7,94 \pm 0.82) \times 10^4$ errechnet werden. Diese Werte stimmen mit den Theorie­werten $\mathcal{F}_{\mathrm{theo}} \approx 19$ sowie $A_{\mathrm{theo}} \approx (35)\times10^4$ nicht überein. 
\vspace{0.3cm}\\
Der Franck–Hertz‐Versuch konnte erfolgreich durchgeführt werden. Es konnte der theoretische Verlauf der mittleren Anregungsenergie $\Delta E$ des vorherrschenden $6{}^1S_0 \to 6{}^3P_1$–Übergangs in Cadmium abhängig von der Temperatur experimentell bestätigt werden. 
\vspace{0.3cm}\\
Weiterhin wurde das dominant angeregte Niveau anhand seiner Anregungsenergien und seines Stoßwirkungsquerschnitts in Übereinstimmung mit dem Experiment ermittelt. Dazu konnte mit $\Delta E = (4{,}83 \pm 0{,}22)\,\mathrm{eV}$  ein Wert errechnet werden, welcher mit dem im Praktikum angegebenen Literaturwert von $4{,}89\,\mathrm{eV}$ übereinstimmt.
\vspace{0.3cm}\\
Dabei gab es nur einen Messwert, welcher unerwartete Ergebnisse lieferte, die mit einer unpassenden Versuchsanordnung zu begründen sind. Weiterhin konnte der theoretische Verlauf der Anodenstromkurven in Abhängigkeit der Gegenspannung $U_g$ bestätigt sowie der Verlauf der Spitzenausschläge in Abhängigkeit der Temperatur erklärt werden. Es konnte ferner präzise erklärt werden, weshalb die Anodenstromkurve stetig verläuft.
