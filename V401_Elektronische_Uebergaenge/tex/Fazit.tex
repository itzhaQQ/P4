\chapter{Fazit}

Dieser Versuch handelte von den elektronischen Übergängen in Atomen und gliedert sich in zwei Hauptteile. Der erste Hauptteil beschäftigt sich mit dem Zeeman-Effekt, während der zweite Teil den Franck-Hertz-Versuch behandelt.
\vspace{0.3cm}\\
Im ersten Versuchsteil konnten wir zunächst optisch mit Hilfe eines Fabry-Pérot-Etalons das Licht einer Cadmium-Lampe abhängig von der Wellenlänge auflösen und mit einem Okular und Polarisationsfiltern die Energieverschiebung durch den Zeeman-Effekt sowohl in transversaler als auch in longitudinaler Richtung beobachten und nachweisen. Im weiteren Versuchsverlauf konnten wir die Energieverschiebung quantitativ mit Hilfe einer CCD-Kamera berechnen und die Abhängigkeit der Energieverschiebung von dem anliegenden Magnetfeld analysieren und darüber das Bohrsche Magneton $\mu_B$ bestimmen. Unsere beiden Werte für $\sigma_+$- und $\sigma_-$-polarisiertes Licht haben eine große Abweichung untereinander gezeigt: $\mu_{\sigma_+} = 8{,}302(77)\times10^{-24}\,\mathrm{J/T}$ und $\mu_{\sigma_-} = 9{,}024(85)\times10^{-24}\,\mathrm{J/T}$. Der Literaturwert hingegen beträgt: $\mu_B = 9{,}2740100783(28)\times10^{-24}\,\mathrm{J/T}$. Unsere Messwerte liegen somit in der richtigen Größenordnung. Der Wert für die $\sigma_-$-Linie stimmt in seiner vierten $\sigma$-Umgebung mit dem Literaturwert überein, dies ist hinsichtlich eines Praktikumsversuchs ein sehr gutes Ergebnis. Der andere Wert hingegen zeigte eine sehr große und unerklärbare Abweichung vom Literaturwert.
\vspace{0.3cm}\\
In weitgehenden Überlegungen konnten für die Finesse $\mathcal{F}$ in transversaler und longitudinaler Konfiguration $\mathcal{F}_{\mathrm{trans}} = (8{,}5 \pm 0{,}4)$ sowie $\mathcal{F}_{\mathrm{long}} = (10{,}07 \pm 0{,}22)$ und für das Auflösungsvermögen $A_{\mathrm{trans}} = (1{,}54 \pm 0{,}08)\times10^5$ sowie $A_{\mathrm{long}} = (1{,}82 \pm 0{,}04)\times10^5$ errechnet werden. Diese Werte stimmen mit den Theorie­werten $\mathcal{F}_{\mathrm{theo}} \approx 19{,}309$ sowie $A_{\mathrm{theo}} = (3{,}493 \pm 0{,}001)\times10^5$ nicht überein. Da jedoch innerhalb der Fehler von $\mathcal{F}_{\mathrm{trans}}$ der Wert $\mathcal{F}_{\mathrm{graph,trans}} = 8{,}65 \pm 0{,}4$ über eine andere Methodik errechnet wurde und Finesse und Auflösungsvermögen gekoppelt sind, ist dies wahrscheinlich auf eine Abweichung der realen Finesse von der theoretischen zurückzuführen und nicht auf eine falsche experimentelle Ermittlung. Weiterhin konnte eine totale Linienbreite $\Delta\lambda_{\mathrm{exp}} = (4 \pm 1)\,\mathrm{pm}$ ermittelt werden, welche innerhalb der relativ großen Fehlergrenzen mit dem erwarteten Wert $\Delta\lambda = (4{,}9 \pm 0{,}2)\,\mathrm{pm}$ übereinstimmt. Somit konnte die Linienverbreiterung durch den Dopplereffekt innerhalb der Fehlergrenzen bestätigt werden.
\vspace{0.3cm}\\
Der Franck–Hertz‐Versuch konnte erfolgreich durchgeführt werden. Es konnte der theoretische Verlauf der mittleren Anregungsenergie $\Delta E$ des vorherrschenden $6{}^1S_0 \to 6{}^3P_1$–Übergangs in Cadmium abhängig von der Temperatur experimentell bestätigt werden. Weiterhin wurde das dominant angeregte Niveau anhand seiner Anregungsenergien und seines Stoßwirkungsquerschnitts in Übereinstimmung mit dem Experiment ermittelt. Dazu konnte mit $\Delta E = (4{,}83 \pm 0{,}22)\,\mathrm{eV}$  ein Wert errechnet werden, welcher mit dem im Praktikum angegebenen Literaturwert von $4{,}89\,\mathrm{eV}$ übereinstimmt. Dabei gab es nur einen Messwert, welcher unerwartete Ergebnisse lieferte, die mit einer unpassenden Versuchsanordnung zu begründen sind. Weiterhin konnte der theoretische Verlauf der Anodenstromkurven in Abhängigkeit der Gegenspannung $U_g$ bestätigt sowie der Verlauf der Spitzenausschläge in Abhängigkeit der Temperatur erklärt werden. Es konnte ferner präzise erklärt werden, weshalb die Anodenstromkurve stetig verläuft.
