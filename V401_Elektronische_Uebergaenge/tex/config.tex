%\documentclass[%
%paper=a4,       % Papiergröße
%fontsize=12pt,  % Schriftgröße
%ngerman         % Option für deutsche Sprache
%]{scrreprt}

% Basics für Codierung und Sprache
% ===========================================================
\usepackage{framed}
\usepackage{lipsum}
\usepackage[final]{graphicx}          % Einbindung von Grafiken
\usepackage{float}                    % für [H]-Option bei Bildern
\graphicspath{{figs/} {images/}}
\usepackage{subcaption}
\usepackage{babel}                    % deutsche Silbentrennung, etc.
\usepackage[german=quotes]{csquotes}  % deutsche Anführungszeichen mit \enquote{...}
\usepackage{titling}
\usepackage{booktabs} % für tabellen
\RequirePackage[backend=biber, style=alphabetic, labelalpha=true]{biblatex}     % erlaubt das Einfügen eines Quellenverzeichnisses
% ===========================================================

% Fonts und Typographie
% ===========================================================
%\usepackage{sourcecodepro}
\usepackage[T1]{fontenc}
%\usepackage{lmodern}
%\usepackage{mathptmx}
\usepackage{mathpazo} % Palatino
%\usepackage{helvet} % Helvetica
%\usepackage{courier} % Courier
%\usepackage{times}
\usepackage{fix-cm}
\usepackage{anyfontsize}
%\usepackage[default]{sourcesanspro}
%\usepackage{nimbusmononarrow}

\usepackage[babel=true,final,tracking=smallcaps]{microtype}
\DisableLigatures{encoding = T1, family = tt* }   % keine Ligaturen für Monospace-Fonts
% ===========================================================

% Farben
% ===========================================================
\usepackage[x11names]{xcolor}
% ===========================================================

% Mathe-Pakete und -Einstellungen
% ===========================================================
%\usepackage{yhmath}
\usepackage{amsmath}
\usepackage{amsthm}
\usepackage{mathtools}             % Tools zum Setzen von Formeln
\usepackage{amssymb,}               % übliche Mathe-Symbole
\usepackage[bigdelims]{newtxmath}  % moderne Mathe-Font
\allowdisplaybreaks{}               % seitenübergreifende Rechnungen
\usepackage{bm}                    % math bold font
%\usepackage{wasysym}               % noch mehr Symbole
\usepackage{siunitx}
\sisetup{output-decimal-marker = {,}} % Kommas als Dezimalteiler
\DeclareSIUnit\barn{b}
% ===========================================================

% TikZ
% ===========================================================
\usepackage{tikz}
\usetikzlibrary{arrows,arrows.meta}   % mehr Pfeile!
\usetikzlibrary{calc}                 % TikZ kann rechnen
\usetikzlibrary{positioning}
\tikzset{>=Latex}                     % Standard-Pfeilspitze
% ===========================================================

% Seitenlayout, Kopf-/Fußzeile
% ===========================================================
\usepackage[bottom=3cm, left=2.5cm, right=2cm]{geometry}
\RequirePackage[headsepline]{scrlayer-scrpage} % edit header and footer of a page, for more lines add [headtopline, headsepline, footsepline, footbotline]
\pagestyle{scrheadings} % set formatting to scrheadings
\clearpairofpagestyles{}
\setkomafont{pageheadfoot}{}

% ===========================================================

% Hyperref für Referenzen und Hyperlinks
% ===========================================================
\usepackage[%
hidelinks,
pdfpagelabels,
bookmarksopen=true,
bookmarksnumbered=true,
linkcolor=black,
urlcolor=SkyBlue2,
plainpages=false,
pagebackref,
citecolor=black,
hypertexnames=true,
pdfborderstyle={/S/U},
linkbordercolor=SkyBlue2,
colorlinks=false,
backref=false]{hyperref}
\hypersetup{final}
\usepackage{cleveref}
\crefname{figure}{Abb.}{Abb.}
\crefname{table}{Tab.}{Tab.}
\crefname{equation}{Gl.}{Gl.}
% ===========================================================

% Listen und Tabellen
% ===========================================================
\usepackage{multicol}
\usepackage[shortlabels]{enumitem}
\setlist{itemsep=0pt}
\setlist[enumerate]{font=\sffamily\bfseries}
\setlist[itemize]{label=$\triangleright$}
\usepackage{tabularx}
% ===========================================================

% listings
% ===========================================================
\usepackage{listingsutf8}
\lstset{
	belowcaptionskip=1\baselineskip,
	breaklines=true,
	showstringspaces=false,
	basicstyle=\ttfamily,
	keywordstyle=\bfseries\color{green!40!black},
	commentstyle=\itshape\color{purple!40!black},
	stringstyle=\color{orange},
	numbers=left,
	numberstyle=\footnotesize\ttfamily\color{gray},
	inputencoding=utf8/latin1,
	tabsize=4,
}

%%%%%%%%%%%%%%%%%%%%%%%%%%%%%%%%%%%%%%%%%%%%%%%%%%%%%%%%%%%
%%% Ab hier folgen nur noch vordefinierte Shortcuts %%%
%%%%%%%%%%%%%%%%%%%%%%%%%%%%%%%%%%%%%%%%%%%%%%%%%%%%%%%%%%%

\newcommand{\BB}{\mathbb{B}}
\newcommand{\CC}{\mathbb{C}}
\newcommand{\NN}{\mathbb{N}}
\newcommand{\QQ}{\mathbb{Q}}
\newcommand{\RR}{\mathbb{R}}
\newcommand{\ZZ}{\mathbb{Z}}
\newcommand{\oh}{\mathcal{O}}

\newcommand{\ol}[1]{\overline{#1}}
\newcommand{\wt}[1]{\widetilde{#1}}
\newcommand{\wh}[1]{\widehat{#1}}

\DeclareMathOperator{\id}{id}                        % Identität
\DeclareMathOperator{\pot}{\mathcal{P}}              % Potenzmenge

% Klammerungen und ähnliches
\DeclarePairedDelimiter{\absolut}{\lvert}{\rvert}    % Betrag
\DeclarePairedDelimiter{\ceiling}{\lceil}{\rceil}    % aufrunden
\DeclarePairedDelimiter{\Floor}{\lfloor}{\rfloor}    % aufrunden
\DeclarePairedDelimiter{\Norm}{\lVert}{\rVert}       % Norm
\DeclarePairedDelimiter{\sprod}{\langle}{\rangle}    % spitze Klammern
%\DeclarePairedDelimiter{\enbrace}{(}{)}              % runde Klammern
\DeclarePairedDelimiter{\benbrace}{\lbrack}{\rbrack} % eckige Klammern
\DeclarePairedDelimiter{\penbrace}{\{}{\}}           % geschweifte Klammern
\newcommand{\Underbrace}[2]{{\underbrace{#1}_{#2}}}  % bessere Unterklammerungen

% Kurzschreibweisen für Faule und Code-Vervollständigung
\newcommand{\abs}[1]{\absolut*{#1}}
\newcommand{\ceil}[1]{\ceiling*{#1}}
\newcommand{\flo}[1]{\Floor*{#1}}
\newcommand{\no}[1]{\Norm*{#1}}
\newcommand{\sk}[1]{\sprod*{#1}}
\newcommand{\enb}[1]{\enbrace*{#1}}
\newcommand{\penb}[1]{\penbrace*{#1}}
\newcommand{\benb}[1]{\benbrace*{#1}}
\newcommand{\stack}[2]{\makebox[1cm][c]{$\stackrel{#1}{#2}$}}

%\newcommand{\vector}[1]{%
%\begin{pmatrix} #1 \end{pmatrix}
%}
%==== Enumerationstyle 1.1.1
\renewcommand{\labelenumii}{\arabic{enumi}.\arabic{enumii}}
\renewcommand{\labelenumiii}{\arabic{enumi}.\arabic{enumii}.\arabic{enumiii}}
\renewcommand{\labelenumiv}{\arabic{enumi}.\arabic{enumii}.\arabic{enumiii}.\arabic{enumiv}}
%

\newcommand{\cby}[1]{\colorbox{yellow}{#1}} % yellow colorbox
\newcommand{\cbb}[1]{\colorbox{black}{#1}} % black colorbox


