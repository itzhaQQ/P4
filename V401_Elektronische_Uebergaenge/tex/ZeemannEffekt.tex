\chapter{Zeemann-Effekt}
Für den Übergang von Cadmium $5^1D_2 \rightarrow 5^1P_1$ wird der normale \texttt{Zeemann}-Effekt untersucht. Dafür folgt zunächst ein theoretischer Einschub.
% -----
\section{Theoretischer Hintergrund}

Der \texttt{Zeemann}-Effekt beschreibt die Aufspaltung von Spektrallinien, und somit Aufhebung der Entartung der Energieniveaus gleicher Gesamtdrehimpulse $J$, in einem externen Magnetfeld. 
Diese Aufspaltung ist auf die Wechselwirkung des magnetischen Moments der Atome mit dem äußeren Magnetfeld zurückzuführen.
Unterschieden wird zwischen dem \textit{normalen} ($S=0$) und dem \textit{anormalen} ($S\neq 0$) \texttt{Zeemann}-Effekt.\\
Da beide Niveaus des Cd einen Gesamtspin von $S=0$ besitzen, ist nur der normale \texttt{Zeemann}-Effekt relevant.
Der Hamiltonian des Elektrons im Atom und im Magnetfeld folgt mit:
\begin{equation}
    \widehat{H}= 
    \underbrace{-\frac{\hbar}{2m}\vec{\nabla}^2 
    -\frac{e^2}{4\pi\varepsilon_0 r}}_{\raisebox{1.5ex}{\scriptsize $H_\mathrm{0}$}} 
    \underbrace{+\beta\frac{\hat{\vec{L}}\cdot\hat{\vec{S}}}{\hbar^2}}_{\raisebox{1.5ex}{\scriptsize $H_{\mathrm{Spin-Bahn}}$}}
    \underbrace{+\mu_B\frac{\hat{\vec{L}}+2\hat{\vec{S}}}{\hbar}\vec{B}}_{\raisebox{1.5ex}{\scriptsize $H_{\mathrm{Zeemann}}$}}
\end{equation}
mit dem \textit{Spin-Bahn}-Kopplungsterm $H_{\mathrm{Spin-Bahn}}$ aus der Feinstruktur und dem \textit{Zeemann}-Kopplungsterm $H_{\mathrm{Zeemann}}$ aus der 
Wechselwirkung mit dem äußeren Magnetfeld mit Richtung $\vec{B}=B\vec{e}_z$.\\
Da jedoch $S=0$ fallen einige Terme weg. Die guten Quantenzahlen, also die, die sich nicht ändern, sind $L$, $S$, $J$ und $M_J$ (wobei $J=L+S=L$) und die 
Energiekorrektur zum Hamiltonian ist:
\begin{equation}
    \Delta E = \mu_B g_J M_J B
\end{equation}
mit dem Landé-Faktor $g_J$:
\begin{equation}
    g_J = 1 + \frac{J(J+1) + S(S+1) - L(L+1)}{2J(J+1)}
\end{equation}
Die Aufspaltung der Energieniveaus ist also linear in $B$ und die Änderung der Energie ist proportional zu $M_J$.\\
% -----
\subsection{Subsection}

\subsubsection{Subsubsection}
